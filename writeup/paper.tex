% TEMPLATE for Usenix papers, specifically to meet requirements of
%  USENIX '05
% originally a template for producing IEEE-format articles using LaTeX.
%   written by Matthew Ward, CS Department, Worcester Polytechnic Institute.
% adapted by David Beazley for his excellent SWIG paper in Proceedings,
%   Tcl 96
% turned into a smartass generic template by De Clarke, with thanks to
%   both the above pioneers
% use at your own risk.  Complaints to /dev/null.
% make it two column with no page numbering, default is 10 point

% Munged by Fred Douglis <douglis@research.att.com> 10/97 to separate
% the .sty file from the LaTeX source template, so that people can
% more easily include the .sty file into an existing document.  Also
% changed to more closely follow the style guidelines as represented
% by the Word sample file. 

% Note that since 2010, USENIX does not require endnotes. If you want
% foot of page notes, don't include the endnotes package in the 
% usepackage command, below.

% This version uses the latex2e styles, not the very ancient 2.09 stuff.
\documentclass[letterpaper,10pt]{article}
\usepackage{epsfig,graphicx,usenix,fullpage,float,hyperref}
\usepackage{mathtools,amssymb,setspace, pbox}
\usepackage[center, labelfont=bf]{caption}

\DeclarePairedDelimiter{\ceil}{\lceil}{\rceil}
\DeclarePairedDelimiter{\floor}{\lfloor}{\rfloor}
\def\bE{\mathbb{E}}

%\usepackage{endnotes}
\begin{document}


%make title bold and 14 pt font (Latex default is non-bold, 16 pt)
\title{\Large \bf 6.824 Final Project}
%for single author (just remove % characters)
\author{
{\rm Colleen Josephson}\\
cjoseph@mit.edu
\and
{\rm Joseph DelPreto}\\
delpreto@mit.ed
\and
{\rm Pranjal Vachaspati}\\
pranjal@mit.edu
\and
{\rm Steven Valdez}\\
dvorak42@mit.edu
% copy the following lines to add more authors
% \and
% {\rm Name}\\
%Name Institution
} % end author

\date{May 11, 2014}

\maketitle

% Use the following at camera-ready time to suppress page numbers.
% Comment it out when you first submit the paper for review.
%\thispagestyle{empty}

\section{Introduction}
This document is the writeup of our 6.824 final project.

The goals of the project are:

\section{Design} \label{sec:design}

\subsection{Paxos Optimizations}

\subsection{Persistent Storage}

\subsubsection{Recovery from failures}

\section{Performance}

We created an extensive system to try and accurately measure performance. In addition to writing many benchmarks, we also modified the original RPC system so that we could run the servers in an AWS cluster and observe the performance on a real network. 

\subsection{Throughput and Latency}
Avg client latency w/ out paxos optimizations:
one shard, 3 groups w/ 3 replicas = 15.4ms/op
many shards, 3 groups w/ 3 replicas = 21.8ms/op

Avg client throughput w/ out paxos optimizations:
one shard, 3 groups w/ 3 replicas = 65 requests/second
many shards, 3 groups w/ 3 replicas = 46 requets/second

Client requests per second:
3 groups w/ 3 replicas:
~179 requests per second (saturates at around 50 users)


\subsection{Bottlenecks}
Although we made significant improvements performance-wise, we identified a number of bottlenecks.

One notable bottleneck is disk writes. The database library we used,
levedb, uses a cache and is quite fast. However, in the process of
debugging networked RPCs, we implemented simple logging. This logging
slowed down the operations by nearly a factor of ten. This was a good
demonstration of how care must be taken in even the most mundane
aspects, such as logging.

\subsubsection{Paxos}

\subsubsection{Disk Recovery}

\section{Conclusion}
We did a project.

%{\footnotesize \bibliographystyle{acm}
%\bibliography{paper}}
%\newpage
\end{document}
